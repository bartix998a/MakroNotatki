\documentclass{article}
\usepackage[utf8]{inputenc}
\usepackage{amsmath}
\usepackage{amsfonts}
\usepackage[polish]{babel}
\usepackage[T1]{fontenc}
\title{Makro}
\author{Bartłomiej Parapura}
\newcommand{\limit}{\lim_{n\to\infty}}
\newcommand{\nrt}[1]{\sqrt[n]{#1}}
\newcommand{\we}[1]{\hat{e}_#1}
\newcommand{\suma}{\sum_{k=0}^n}
\newcommand{\harr}{\Leftrightarrow}
\newcommand{\rarr}{\Rightarrow}
\newcommand{\larr}{\Leftarrow}
\newcommand{\rcw}{\mathbb{R}}
\newcommand{\istnieje}[1]{\exists_{#1}\;}
\newcommand{\s}{\;}
\newcommand{\lin}[1]{\text{lin}\{#1\}}
\begin{document}
\maketitle
\section{Oznaczenia}
\begin{enumerate}
    \item \(Y\) - Funkcja produkcji lub całkowite PKB.
    \item \(C\) - Konsumpcja. Wydatki na konsumpcje.
    \item \(G\) - wydatki rządowe bez transferów.
    \item \(I\) - Oszczędności prywatne.
    \item \(NX\) - Netto export
    \item \(S\) - Oszczędności narodowe.
    \item \(P\) - Deflator PKB.
    \item \(\varepsilon\) - Realny kurs walutowy.
    \item \(V\) - Prędkość opbiegu pieniądza
    \item \(M\) - Zasób pieniądza
\end{enumerate}
Oszcędnosci:
\begin{gather}
    S = Y - C - G
\end{gather}
Uogolnienie produkcji dla otwartej gospodarki\footnote{dla zamknietej \(NX = 0\)} oraz jakas pierdola z \(NX\):
\begin{gather}
    Y = C + I + G + NX\\
    NX = S - I\\
    NX = \overline{S} - I(r^*)
\end{gather}
To 3 jest prawdą bo produkcja podatki i wydatki rządowe są egzogeniczne. 
Krajowa stopa procentowa zależy od oszczędności.
\end{document}
