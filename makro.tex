\documentclass{article}
\usepackage[utf8]{inputenc}
\usepackage{amsmath}
\usepackage{amsfonts}
\usepackage[polish]{babel}
\usepackage[T1]{fontenc}
\title{Makro}
\author{Bartłomiej Parapura}
\newcommand{\limit}{\lim_{n\to\infty}}
\newcommand{\nrt}[1]{\sqrt[n]{#1}}
\newcommand{\we}[1]{\hat{e}_#1}
\newcommand{\suma}{\sum_{k=0}^n}
\newcommand{\harr}{\Leftrightarrow}
\newcommand{\rarr}{\Rightarrow}
\newcommand{\larr}{\Leftarrow}
\newcommand{\rcw}{\mathbb{R}}
\newcommand{\istnieje}[1]{\exists_{#1}\;}
\newcommand{\s}{\;}
\newcommand{\lin}[1]{\text{lin}\{#1\}}
\begin{document}
\maketitle
\section{Oznaczenia}
\begin{enumerate}
    \item \(Y\) - Funkcja produkcji lub całkowite PKB.
    \item \(C\) - Konsumpcja. Wydatki na konsumpcje.
    \item \(G\) - wydatki rządowe bez transferów.
    \item \(I\) - Oszczędności prywatne.
    \item \(NX\) - Netto export
    \item \(S\) - Oszczędności narodowe.
    \item \(P\) - Deflator PKB.
    \item \(\varepsilon\) - Realny kurs walutowy.
    \item \(V\) - Prędkość obiegu pieniądza
    \item \(M\) - Zasób pieniądza
    \item \(M0\) - baza monetarna
    \item \(M1\) - podaż pieniądza\footnote{zasadniczo to samo co \(M\) tylko tego się użwa przy mnożniku}
    \item \(r\) - realna stopa procentowa
    \item \(i\) - nominalna stopa procentowa
    \item \(\pi\) - inflacja \footnote{Kto wymyślił to oznaczenie!!!}
    \item \(L\) - popyt na pieniądz
    \item \(W\) - płaca nominalna
    \item \(R\) - dochód dla kapitału
    \item \(PE\) - planowane wydatki\footnote{Oznaczenie z mankiwa}
    \item \(*\) - Operator mówiący, że cos jest światowe
    \item \(\overline{\phantom{a}}\) - Operator mówiący, że coś jest dane z góry
    \item \(E\) - Operator mówiący, że coś jest przeiwdywane.
\end{enumerate}
Oszcędnosci:
\begin{gather}
    S = Y - C - G
\end{gather}
\section{klasyczna zamknięta gospodarka}
\begin{gather}
    \frac{\partial Y}{\partial L} = \frac{W}{P}\\
    \frac{\partial Y}{\partial K} = \frac{K}{P}
\end{gather}
Z tego oraz równania YCIG oraz dodatkowych równań z treści wyjdzie wszystko. 
Należy dodać, że funkcja produkcji zależy od popytu na pracę i kapitał.
\section{Pieniądz i inflacja}
Równanie z prędkością obiegu:
\begin{gather}
    MV = PY
\end{gather}
Definicja inflacji w okresie \((t-1; t)\):
\begin{gather}
    \pi = \frac{P_{t}}{P_{t-1}} - 1
\end{gather}
Zależność między realna stopą procentową nominalną a inflacją:
\begin{gather}
    1 + r = \frac{1+i}{1+\pi}
\end{gather}
Lub w przybliżeniu w którym działąmy:
\begin{gather}
    r = i -\pi
\end{gather}
Ale że inflacje trzeba zgadywać:\footnote{chyba, szczerze nie mam pojęcia}
\begin{gather}
    i = r + E\pi
\end{gather}
Definicja popytu na pieniądz:
\begin{gather}
    \frac{M}{P} = L
\end{gather}
\subsection{podaż pieniądza}
Mamy agregaty \(M0 = C + R\)\footnote{tutaj \(C\) to jest gotowka w obiegu, \(D\) - suma depozytów w bankach
prywatnych oraz \(R\) suma rezerw w bankach prywatnych} oraz \(M1 = C + D\). Mnożnik się definiuje tak:
\begin{gather}
    m=\frac{M1}{M0} = \frac{C+D}{C + R}
\end{gather}
A jak cała gotówka pójdzie do banków to mamy tak:
\begin{gather}
    m = \frac{1}{r}
\end{gather}
Gdzie to \(r\) to tym razem stopa rezerw które trzymają banki.
\section{klasyczna gospodarka otwarta}
Uogolnienie produkcji dla otwartej gospodarki\footnote{dla zamknietej \(NX = 0\)} oraz jakas pierdola z \(NX\):
\begin{gather}
    Y = C + I + G + NX\\
    NX = S - I\\
    NX = \overline{S} - I(r^*)
\end{gather}
Krajowa stopa procentowa zależy od oszczędności. Zasadniczo działą chyba tak jak w zamkniętej.
Kurs walutowy:
\begin{gather}
    \varepsilon = \frac{P}{P^*}
\end{gather}
\section{Bezrobocie}
Niech \(L\) będzie liczbą ludzi, którzy mogą pracować. \(E\) to będzie
liczba ludzi zatrudnionych oraz \(U\) liczba ludzi bezrobotnych. 
Stopa bezrtobocia to \(\frac{U}{U+E} = \frac{U}{L}\). Ponadto niech:
\begin{enumerate}
    \item \(s\) będzie odsetkiem ludzi zatrudnionych tracących prace w danej jednostce czasu.
    \item \(f\) będzie odsetkiem ludzi bezrobotnych znajdujących prace w danej jednostce czasu.
\end{enumerate}
Dla sytuacji gdzie bezrobocie jest stałe:
\begin{gather}
    \frac{U}{L} = \frac{s}{s+f}
\end{gather}
\section{model IS-LM}
Definiujemy sobie kolejną już zmienną pomocniczą, nazywamy to planowane wydatki:
\begin{gather}
    PE = C + I + G
\end{gather}
W równowadze mamy:
\begin{gather}
    Y = PE
\end{gather}
\subsection{Krzywa IS}
Krzywa IS będzie wykresem funkcji \(Y(r)\) lub \(r(Y)\) wziętej z równania:\footnote{tutaj wydatki = dochód}
\begin{gather}
    Y = C(Y - T) + I(r) + G
\end{gather}
Gdzie oczyiscie jak coś jest z nawiasmai to znaczy, że to funkcja czegoś innego
którą dostaniemy z góry na kolosie. 
\subsection{Krzywa LM}
To samo co tam tylko bierzemy z równania:\footnote{tutaj popyt na pieniądz = podaż pieniądza}    
\begin{gather}
    \frac{M}{P} = L(r,Y)
\end{gather}
Punkt równowagi w gospodarce jest tam gdzie nasze krzywe się przecinają.
\end{document}
